\section*{2017}
\vspace{-.5cm}
\hrulefill \smallskip\\
\ques{7}{a}{15} Given below are the average wage in Rupees pre hour of unskilled workers of a factory during the period 2005-2010 and corresponding consumer price index numbers taken 2005 as base year with price index equal to 100. Determine the real wages of the workers during 2005-2010 compared with the wages in 2005.
\begin{center}
    \begin{tabular}{|*{7}{c|}}\hline
        Year & 2005 & 2006 & 2007 & 2008 & 2009 & 2010 \\ \hline
        Average wage per hour (Rs.) & 11.9 & 19.4 & 21.3 & 22.8 & 24.5 & 31.0 \\
        Consumer Price Index & 100 & 120.2 & 121.7 & 125.9 & 129.2 & 140\\\hline
    \end{tabular}
\end{center} Also find the worth of one Rupee in each subsequent year compared to one Rupee in 2005.
\myline
\ques{8}{c}{15} Define Index numbers. Give its uses and limitations.