\section*{2009}
\vspace{-.5cm}
\hrulefill \smallskip\\
\ques{6}{b}{20} What do yo understand by cost of living index number? explain how would you construct it. \\ The following table gives the group index numbers and weights of different heads in the calculation of cost of living index number except the index for the group `fuel and lighting':
 \begin{center}
    \begin{tabular}{l*{2}{c}}\hline
    Group & Index & Weight \\ \hline
    Food & 221 & 35\\
    Clothing & 198 & 14 \\
    Fuel and Lighting & -- & 15 \\
    Rent & 183 & 8 \\
    Miscellaneous & 161 & 20 \\ \hline
    \end{tabular}
\end{center} If the overall cost of living index is 193, find the index number of the fuel and lighting group.
\myline
\ques{7}{b}{20} What are the tests to be satisfied by a good index number? Examine how far they are met using Paasche's index number and Fischer's ideal index number.\\Using Laspeyres' and Paasche's index numbers, the price index numbers for the year 1970 with 1966 as base year are 121 and 144 respectively. What would be the value of Fisher's index number?