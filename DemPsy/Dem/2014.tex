\section*{2014}
\vspace{-.5cm}
\hrulefill \smallskip\\
\ques{5}{e}{10} Why the gross reproduction rate (GRR) and net reproduction rate (NRR) are considered refined measures of fertility?\\
Assuming that the ration of female babies to total birth is 48.8\%, copmute the gross reproduction rate for the following data:
\begin{center}
    \resizebox{13cm}{!}{
    \begin{tabular}{|p{2.5cm}| *{7}{c|}}
    \hline
    Age Group & 16-20 & 21-25 & 26-30 & 31-35 & 36-40 & 41-45 & 46-50\\ \hline
    Fertility rate per thousand women & 19 & 173 & 253 &201 & 157 & 67 & 9 \\ \hline

    \end{tabular}}
\end{center}
\ques{6}{a}{20} Fill in the blanks in a portion of the life table given below:
\begin{center}
    \begin{tabular}{| c | *{6}{c|} c|}
    \hline
    \emph{Age in Years} & $l_x$ & $d_x$ & $q_x$ & $P_x$ & $L_x$ & $T_x$ & $E_x^\circ$ \\ \hline
    4  & 95000 & 500 & ? & ? & ? & 4850300 & ?\\ \hline 
    5 & ?     & 400 & ? & ? & ? & ? & ?\\ 
    \hline
    \end{tabular}
\end{center}
\ques{6}{c}{15} Given below is the data regarding deaths in two districts. In the basis of the given data, calculate the standardized death rates. Give your comments.
\begin{center}
    \begin{tabular}{| c | *{4}{c|} c|}
    \hline
    \multirow{2}{*}{ Age Range}&\multicolumn{2}{c|}{Population (0 0)}&\multicolumn{2}{c|}{Number of deaths (0 0)} & \multirow{2}{3cm}{Age Distribution of Standard 1000} \\ \cline{2-5}
    & District A & District B & District A & District B & \\ \hline
    0-10 & 2000 & 1000 & 50 & 20 & 206 \\ \hline
    10-55 &7000 & 3000 & 75 & 30 & 583 \\ \hline
    55 and above& 1000 & 2000 & 25 & 40 & 211 \\ 
    \hline
    \end{tabular}
\end{center}
\ques{8}{b}{20} What do you mean by stable and quasi-stable populations? Stating the basic assumptions of stable population theory, derive three basic equations which provide information about the intrinsic growth rate, birth rate and age distribution of population.
