\section*{2009}
\vspace{-.5cm}
\hrulefill \smallskip\\
\ques{5}{c}{12} Differentiate between total fertility rate (TFR), and gross reproduction rate (GRR) and net reproduction rate (NRR). Does TFR strictly conform to our idea of a measure for reproduction? How does NRR indicate the growth of population? 
\myline
\ques{5}{e}{12} What do you mean by a life table?\\ The values of $l_x$ in a life table are given as follows:
\begin{center}
    \begin{tabular}{*{8}{c}}
    age ($x$) : & 102 & 103 & 104 & 105 & 106 & 107 & 108 \\
    $l_x$ : & 97 & 59 & 32 & 15 & 6 & 2 & 0
    \end{tabular}
\end{center} Calculate remaining entries of the life table for $x\geq102$ and find the probability that a person of exact age 102 years will die between ages 103 and 107. 
\myline
\ques{7}{a}{25} Discuss the differences between direct and indirect methods of standardization of death rates. \\ Calculate the crude an standardized death rate of the year 1990 by direct and indirect methods of standardization by taking 1980 population as standard:
\begin{center}
    \begin{tabular}{|p{2cm}|cc|cp{2.5cm}|}\hline
    \multirow{2}{}{Age group (years)} & \multicolumn{2}{c|}{1980} &\multicolumn{2}{c|}{1990}\\ \cline{2-5}
    & Population & Total Deaths & Population & Age Specific death rates per 1000 \\ \hline
    0-10 & 10000 & 240 & 15000 & 22.0 \\
    10-25 & 12000 & 145 & 15000 & 13.0 \\
    25-60 & 6000 & 92 & 8000 & 15.0 \\
    60+ & 8000 & 480 & 9000 & 47.0 \\ \hline
    \end{tabular}
\end{center}
\myline
\ques{8}{a}{20} Prove the following :
{\large
\begin{enumerate}[label=(\roman*)]
\itemsep0em 
\item $ \displaystyle p_x = \: \frac{e_x}{1 + e_{x+1}}$
\item $ \displaystyle m_x = \:\mu_{x+\frac{1}{2}}$
\item $\displaystyle \mu_x \simeq \:\frac{8(l_{x-1} - l_{x+1}) - (l_{x-2} -l_{x+2})}{12l_x}$
\end{enumerate}}
