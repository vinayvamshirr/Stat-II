\section*{2015}
\vspace{-.5cm}
\hrulefill \smallskip\\
\ques{5}{c}{10} Explain different (i) methods of collecting census data and (ii) types of error in census data.
\myline
\ques{5}{d}{10} Discuss the registration of vital statistics in India, stating its uses and limitations.
\myline
\ques{7}{a}{15} Compute Age Specific Fertility (ASFR) and Total Fertility Rate(TFR) from the following table:
\begin{center}
    %\resizebox{8cm}{!}{
    \begin{tabular}{| c | c | c|} \hline
    Age group (in years) &  Female population ('000) & No. of live births\\ \hline
    15-19 & 200.6 & 4227  \\ 
    20-24 & 173.5 & 26099\\ 
    25-29 & 161.7 & 32844 \\ 
    30-34 & 160.9 & 23449 \\ 
    35-39 & 155.7 & 11588  \\ 
    40-44 & 125.6 & 2071  \\ 
    45-49 & 87.6 & 122 \\ \hline
    \end{tabular}
\end{center}
\ques{7}{b}{15} Explain the method of population projection using logistic curve. State its limitations.
\myline
\ques{8}{b}{20} The following table gives the number of female births classified by age of mothers and survival rates of mothers :
\begin{center}
    %\resizebox{8cm}{!}{
    \begin{tabular}{| c | c | c| c |} \hline
    Age of mothers  &  Female population & No. of female & Survival Rate \vspace{-1.5ex}\\ 
    (in years) &  ('00) & live births ('00) & (per 100000) \\ \hline
    15-19 & 157670 & 4632 & 58065  \\ 
    20-24 & 147624 & 14443 & 55870\\ 
    25-29 & 124200 & 14058 & 52981\\ 
    30-34 & 105865 & 8329  & 48963\\ 
    35-39 & 89264 & 4036  & 44146 \\ 
    40-44 & 77887 & 2158 & 39154\\ 
    45-49 & 61161 & 689 & 34198\\ \hline
    \end{tabular}
\end{center} Compute Gross Reproduction Rate (GRR) and Net Reproduction Rate (NRR) and draw your conclusion.