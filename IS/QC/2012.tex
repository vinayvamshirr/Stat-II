\section*{2012}
\vspace{-.5cm}
\hrulefill \smallskip\\
\ques{1}{a}{12} Describe single and double sampling plans used in acceptance sampling. Define the operating characteristic function of a sampling plan.
\myline
\ques{1}{c}{12} Describe how control chart based on sample ranges is constructed.
\myline
\ques{3}{a}{30} A sample of 100 screws was selected on each the 25 successive days in a factory manufacturing screws, and each screw was examined for defects. The data for number of unacceptable screws on different days are given below:
\begin{center}
    \resizebox{8cm}{!}{
    \begin{tabular}{| c | c | c| c|}
    \hline
    Day &  \# of defective & Day &  \# of defective \\ \hline
    1 & 7 & 14 & 6 \\ \hline
    2 & 4 & 15 & 2 \\ \hline
    3 & 3 & 16 & 9 \\ \hline
    4 & 6 & 17 & 7 \\ \hline
    5 & 4 & 18 & 6 \\ \hline
    6 & 9 & 19 & 7 \\ \hline
    7 & 6 & 20 & 11 \\ \hline
    8 & 7 & 21 & 6 \\ \hline
    9 & 5 & 22 & 7 \\ \hline
    10 & 3 & 23 & 4 \\ \hline
    11 & 7 & 24 & 8 \\ \hline
    12 & 8 & 25 & 6 \\ \hline
    13 & 4   \\  \cline{1-2}
    \end{tabular}}
\end{center}
Assume that the production process was in control during this period. Determine the upper and the lower control limits(UCL and LCL) based on this data for the proportion of defective items.
\myline
\ques{3}{b}{30} Consider an acceptance sampling plan where 50 items sampled from a hug lot will be examined and the lot will be accepted if at most two of the sampled items are found defective - otherwise the lot will be rejected. Evaluate the acceptance probability as a function of the proportion of defectives in the lot and sketch th Operating Characteristic (OC) curve. (Only a rough sketch is required and no graph paper necessary)




